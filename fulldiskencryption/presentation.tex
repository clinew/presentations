% Thanks to Kevin D. McGrath for the presentation template.
\documentclass[xcolor={dvipsnames,svgnames},hyperref=dvips]{beamer}

% Packages.
\usepackage{graphicx}
\usepackage{amssymb}
\usepackage{amsmath}
\usepackage{amsthm}
\usepackage{url}
\usepackage{pstricks}
\usepackage{pst-node}

% Hyperlinks.
\def\name{Wade Cline}
\hypersetup{
  colorlinks = true,
  citecolor = black,
  linkcolor = black,
  urlcolor = black,
  pdfauthor = {\name},
  pdfkeywords = {``computer science'', lug, cryptography, full-disk},
  pdftitle = {Linux Full-Disk Encryption},
  pdfsubject = {Full-Disk Encryption Basics},
  pdfpagemode = UseNone
}

% ???
\usetheme[hideothersubsections]{Hannover}
\usecolortheme{sidebartab}

% Presentation metadata.
\title[Full-Disk Encryption]{}
\author{Wade Cline}
\date{04 March 2014}

\AtBeginSection[]
{
  \begin{frame}<beamer>{Outline}
    \tableofcontents[currentsection]
  \end{frame}
}
\AtBeginSubsection[]
{
  \begin{frame}<beamer>{Outline}
    % \transwipe
    \tableofcontents[currentsection,currentsubsection]
  \end{frame}
}

\begin{document}

% Title.
\begin{frame}
  \titlepage
\end{frame}

% Table of Contents.
\begin{frame}{Outline}
  % \transwipe
  \tableofcontents
  % You might wish to add the option [pausesections]
\end{frame}

\section{Basics}\label{section:basics}
	\subsection{Plaintext vs. Ciphertext}
	\begin{frame}
		\frametitle{Plaintext vs. Ciphertext}
		\begin{itemize}
		\item Plaintext: Unencrypted data.
		\item Ciphertext: Encrypted data.
		%TODO: Encryption and Decryption images.
		\end{itemize}
	\end{frame}

	\subsection{Symmetric vs. Asymetric Cryptography}
	\begin{frame}
		\frametitle{Symmetric Cryptography}
		\begin{itemize}
		\item Same key used to encrypt plaintext and to decrypt ciphertext. 
		\item Useful when only one person needs access to information.
		\item Subject of today's discussion.
		\item Examples: AES, Blowfish, 3DES, Serpent, Twofish.
		\end{itemize}
	\end{frame}

	\begin{frame}
		\frametitle{Asymetric Cryptography}
		\begin{itemize}
		\item AKA Public-Private Key Cryptography.
		\item Different keys used to encrypt plaintext and to decrypt ciphertext.
		\item Useful when communicating securely between two people.
		\item Example: RSA.
		\end{itemize}
	\end{frame}

	\subsection{Stream vs. Block Ciphers}
	\begin{frame}
		\frametitle{Stream vs. Block Ciphers}
		\begin{itemize}
		\item Block Ciphers operate on larged, fixed-length chunks of bits.
		\item Stream Ciphers encrypt digits one-at-a-time.
		\item Similar to Block vs. Character Devices.
		\item Block Ciphers subject of today's discussion.
		\end{itemize}
	\end{frame}

	\subsection{Disk Layout}
	\begin{frame}
		\frametitle{Disk Layout}
		\begin{itemize}
		\item Linux divides disks into 512-byte sections known as \textit{sectors}.
		\item Filesystem divides read/writes into \textit{blocksize} units (multiple of sector size).
		% TODO: Visualization.
		\end{itemize}
	\end{frame}

	\subsection{Wear-leveling}
	\begin{frame}
		\frametitle{Wear-leveling}
		\begin{itemize}
		\item Attempts to distribute writes evenly across the device.
		\item Helps with device longevity.
		\item Used mostly in USB and SSD devices.
		\item Makes overwriting files difficult.
		\end{itemize}
	\end{frame}

\section{Wiping}\label{wiping}
	\subsection{Naive Wiping}
	\begin{frame}
		\frametitle{Naive Wiping}
		\begin{itemize}
		\item Nuke disk \textit{once} with a series of '0's.
		\item Fast, simple, but leaks data.
		\item Each filesystem has a metaphorical ``fingerprint''.
		\item Proper analysis may reveal old data.
		% TODO: Visualization/examples.
		\end{itemize}
	\end{frame}

	\subsection{Secure Wiping}
	\begin{frame}
		\frametitle{Secure Wiping}
		\begin{itemize}
		\item Use cryptographic-grade pseudo-random numbers.
			\begin{itemize}
			\item \texttt{/dev/urandom} should provide a decent stream.
			\item \texttt{/dev/random} in theory better, but in practice you'd be dead before completion.
			\end{itemize}
		\item Wipe multiple times.
			\begin{itemize}
			\item In theory, proper analysis could reveal previous magnetic states.
			\item Government-recommended is 7 times.
			\end{itemize}
		\item Depending on size of drive, could take days or weeks.
		\end{itemize}
	\end{frame}

\section{Encryption}\label{section:encryption}
	\subsection{Mappings}
	\begin{frame}
		\frametitle{Mappings}
		\begin{itemize}
		\item Rather than ``encrypt'' function, create abstraction layer over device.
		\item Entries go under the \texttt{/dev/mapper} directory.
		\item For example, use \texttt{/dev/mapper/root} to access an encrypted \texttt{/dev/sda}.
			\begin{itemize}
			\item \texttt{/dev/sda} appears as garbage (because it's encrypted) to anyone looking at it.
			\item \texttt{/dev/mapper/root} looks like a normal hard drive.
			\end{itemize}
		% TODO: Visualization.
		\end{itemize}
	\end{frame}

	\subsection{\texttt{dm-crypt} vs. LUKS}
	\begin{frame}
		\frametitle{\texttt{cryptsetup}}
		\begin{itemize}
		\item \texttt{cryptsetup} front-end to kernel crypto API.
		\item Two main modes: ``plain'' \texttt{dm-crypt} and LUKS (Linux Unified Key Setup).
		\item LUKS is feature-rich, \texttt{dm-crypt} is not.
		\item Detailed description of LUKS beyond scope of presentation.
		\item Simplfied: Beginners use LUKS, advanced users \textit{may} wish to use \texttt{dm-crypt}.
		\end{itemize}
	\end{frame}

	\subsection{Ciphers}
	\begin{frame}
	\end{frame}

	\subsection{Chaining}
	\begin{frame}
	\end{frame}

	\subsection{Initialization Vector}
	\begin{frame}
	\end{frame}

	\subsection{Keyfiles}
	\begin{frame}
	\end{frame}

	\subsection{Multiple Layes}
	\begin{frame}
	\end{frame}

\section{Early Userspace}\label{section:hell}
	\subsection{Early-Userspace hell}
	\begin{frame}
	\end{frame}

% References?
\begin{frame}
\end{frame}

\end{document}

